%----------------------------------------------------------------------
\chapter{搜索器需求说明}
%----------------------------------------------------------------------
\section{任务说明}
\subsection{目标}
\begin{enumerate}
\item 对当前流行的全文搜索引擎进行认真的研究,以了解其主要工作流程,方便后期进行系统的设计与开发工作。
\item 高度重视系统的分析与设计工作,保证系统有较高的可扩展性和安全性。
\item 系统使用C语言、Python语言等多种语言合作开发,在保证系统可高效、稳定运行的同时要充分发挥不同类型语言的优势,另外,还就注意处理好不同语言间的数据交互问题。
\item 搜索器应该包含网络爬虫和网页链接提取的功能,可实现自动化地下载网页、提取链接的工作。
\end{enumerate}
\subsection{运行环境}
\begin{description}
\item[操作系统]Linux2.6
\item[数据库]mysql5.5
\item[Python]python2.7
\item[CPU]intel-i386
\end{description}
\subsection{条件与限制}
\begin{enumerate}
\item 开发、测试与运行环境均处于校园网内,网络环境相对简单,并且网速较快,不会成为系统的限制因素。
\item 有较多的分析与设计案例可以参考。
\item 开发与测试的主机硬件条件(CPU频率、内存容量、硬盘容量等)较差,难以满足系统高负荷运行的需要。
\item 开发者本身对开发语言等并不足够熟悉。
\end{enumerate}

\section{数据描述}


\section{功能需求}
搜索器包含网络爬虫与链接提取的功能。从初始链接开始,在配置文件和数据库中相关信息的辅助与限制下,可以自动地进行网页下载以及从网页源码中提取链接的工作。
\subsection{网络爬虫}
\begin{enumerate}
\item 可从初始链接开始自动到校园网中抓取网页。
\item 能够识别链接的权重,并先下载权重较高的网页。
\item 可以处理重复的网页,以及链接相同但内容不同的网页
\end{enumerate}
\subsection{网页链接提取}
\begin{enumerate}
\item 不断地对新出现的网页进行分析,提取其中的链接。
\item 可提取以不同形式出现的链接,如html标签<a>中出现的链接、javascript重定向的链接等。
\item 具有计算链接权重的能力。
\end{enumerate}

