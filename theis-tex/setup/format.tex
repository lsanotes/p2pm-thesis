%%%%%%%%%%%%%%%%%%%%%%%%%%%%%%%%%%%%%%%%%%%%%%%%%%%%%%%%%%%%%%%%%%%%%%%%%
%
%   LaTeX File for Bachelor Thesis of 
%               Chongqing University of Posts and Telecommunications
%   TeXLive     重庆邮电大学    本科毕业论文模板
%   Based on Lei Wang 's Template for Tsinghua Uniersity
%   Version: 1.00
%   Last Update: 2011-5-17
%
%%%%%%%%%%%%%%%%%%%%%%%%%%%%%%%%%%%%%%%%%%%%%%%%%%%%%%%%%%%%%%%%%%%%%%%%%
%   Copyright 2002       by  Wang Tianshu               (tswang@asia.com)
%   Copyright 2002-2003  by  Lei Wang (BaconChina)       (bcpub@sina.com)
%%%%%%%%%%%%%%%%%%%%%%%%%%%%%%%%%%%%%%%%%%%%%%%%%%%%%%%%%%%%%%%%%%%%%%%%%
%
%%%%%%%%%%%%%%%%%%%%%%%%%%%%%%%%%%%%%%%%%%%%%%%%%%%%%%%%%%%%%%%%%%%%%%%%%
%   Copyright 2011-2012  by  Zhang Ao (Able Zhang)    (ao.2322@gmail.com)
%%%%%%%%%%%%%%%%%%%%%%%%%%%%%%%%%%%%%%%%%%%%%%%%%%%%%%%%%%%%%%%%%%%%%%%%%


%%%%%%%%%%%%%%%%%%%%%%%%%%%%%%%%%%%%%%%%%%%%%%%%%%%%%%%%%%%
%
%   主文档格式定义
%
%%%%%%%%%%%%%%%%%%%%%%%%%%%%%%%%%%%%%%%%%%%%%%%%%%%%%%%%%%%

\begin{CJK*}{UTF8}{song}
\end{CJK*}

%% 按清华标准, 将版芯控制在240mm以内, 正文范围控制在220mm以内
%%\addtolength{\headsep}{-0.1cm}          %页眉位置
%%\addtolength{\footskip}{-0.1cm}         %页脚位置
%\addtolength{\topmargin}{0.5cm}
%
%%%%%%%%%%%%%%%%%%%%%%%%%%%%%%%%%%%%%%%%%%%%%%%%%%%%%%%%%%%%
%% 公式的精调
%%%%%%%%%%%%%%%%%%%%%%%%%%%%%%%%%%%%%%%%%%%%%%%%%%%%%%%%%%%%
%
%%\setlength{\mathindent}{4.7 em}     %左对齐公式缩进量
%
%% \eqnarray如果很长,影响分栏、换行和分页(整块挪动,造成页面空白),
%% 可以设置成为自动调整模式
%\allowdisplaybreaks[4]
%
%%%%%%%%%%%%%%%%%%%%%%%%%%%%%%%%%%%%%%%%%%%%%%%%%%%%%%%%%%%
%下面这组命令使浮动对象的缺省值稍微宽松一点,从而防止幅度
%对象占据过多的文本页面,也可以防止在很大空白的浮动页上放置
%很小的图形。
%%%%%%%%%%%%%%%%%%%%%%%%%%%%%%%%%%%%%%%%%%%%%%%%%%%%%%%%%%%
\renewcommand{\textfraction}{0.15}
\renewcommand{\topfraction}{0.85}
\renewcommand{\bottomfraction}{0.65}
\renewcommand{\floatpagefraction}{0.60}


%%%%%%%%%%%%%%%%%%%%%%%%%%%%%%%%%%%%%%%%%%%%%%%%%%%%%%%%%%%%
%%下面这组命令可以使公式编号随着每开始新的一节而重新开始。
%%%%%%%%%%%%%%%%%%%%%%%%%%%%%%%%%%%%%%%%%%%%%%%%%%%%%%%%%%%%
%
%%\makeatletter      % '@' is now a normail "letter" for TeX
%%\@addtoreset{eqation}{section}
%%\makeatother       % '@' is restored as a "non-letter" character for TeX

%%%%%%%%%%%%%%%%%%%%%%%%%%%%%%%%%%%%%%%%%%%%%%%%%%%%%%%%%%%%
% 重定义字体命令
%%%%%%%%%%%%%%%%%%%%%%%%%%%%%%%%%%%%%%%%%%%%%%%%%%%%%%%%%%%%

%\setmainfont{TeX Gyre Termes}
%\setmainfont{Latin Modern Roman}
%%\setsansfont{TeX Gyre Heros}
%\setmonofont{Latin Modern Mono}
%%\setCJKmainfont[BoldFont={方正小标宋简体}]{方正书宋简体}    % 宋体  
%\setCJKmainfont[BoldFont={方正小标宋简体}]{Adobe Song Std}    % 宋体  
%\setCJKsansfont{Adobe Heiti Std}
%\setCJKmonofont{Adobe Fangsong Std}
%
%%\setCJKfamilyfont{song}[BoldFont={方正宋黑简体}]{SimSun}      	% 宋体  
%%\setCJKfamilyfont{song}[BoldFont={方正宋三_GBK}]{方正博雅宋_GBK}  % 宋体  
%%\setCJKfamilyfont{song}[BoldFont={Adobe Heiti Std}]{Adobe Song Std}    % 宋体  
%%\setCJKfamilyfont{song}[BoldFont={华文中宋}]{华文宋体}    % 宋体  
%%\setCJKfamilyfont{song}[BoldFont={方正大标宋_GBK}]{方正兰亭宋_GBK}    % 宋体  
%%\setCJKfamilyfont{song}[BoldFont={方正小标宋简体}]{方正书宋简体}    % 宋体  
%%文泉驿微米黑
%\setCJKfamilyfont{song}[BoldFont={方正小标宋简体}]{Adobe Song Std}    % 宋体  
%\setCJKfamilyfont{hei}{Adobe Heiti Std}      	% 黑体  
%\setCJKfamilyfont{kai}{AR PL UKai TW MBE}      	% 楷体  
%%\setCJKfamilyfont{fang}{Adobe Fangsong Std}  	% 仿宋体
%\setCJKfamilyfont{nwpulogo}{nwpulogo}        	% 含"西北工业大学"logo字体 

\newcommand{\song}{\CJKfamily{song}}
\newcommand{\hei}{\CJKfamily{hei}}
\newcommand{\fang}{\CJKfamily{fang}}
\newcommand{\kai}{\CJKfamily{kai}}

%%%%%%%%%%%%%%%%%%%%%%%%%%%%%%%%%%%%%%%%%%%%%%%%%%%%%%%%%%%
% 重定义字号命令
%%%%%%%%%%%%%%%%%%%%%%%%%%%%%%%%%%%%%%%%%%%%%%%%%%%%%%%%%%%

\newcommand{\chuhao}{\fontsize{42pt}{63pt}\selectfont}    % 初号, 1.5倍行距
\newcommand{\yihao}{\fontsize{26pt}{36pt}\selectfont}    % 一号, 1.4倍行距
\newcommand{\xiaoyi}{\fontsize{24pt}{32pt}\selectfont}    % 一号, 1.4倍行距
\newcommand{\erhao}{\fontsize{22pt}{28pt}\selectfont}    % 二号, 1.25倍行距
\newcommand{\xiaoer}{\fontsize{18pt}{18pt}\selectfont}    % 小二, 单倍行距
\newcommand{\sanhao}{\fontsize{16pt}{24pt}\selectfont}    % 三号, 1.5倍行距
\newcommand{\xiaosan}{\fontsize{15pt}{22pt}\selectfont}    % 小三, 1.5倍行距
\newcommand{\sihao}{\fontsize{14pt}{21pt}\selectfont}    % 四号, 1.5倍行距
\newcommand{\banxiaosi}{\fontsize{13pt}{16.25pt}\selectfont}    % 半小四, 1.25倍行距
\newcommand{\xiaosi}{\fontsize{12.5pt}{12.5pt}\selectfont}    % 小四, 1.2倍行距
\newcommand{\dawuhao}{\fontsize{11pt}{11pt}\selectfont}    % 大五号, 单倍行距
\newcommand{\wuhao}{\fontsize{10.5pt}{10.5pt}\selectfont}    % 五号, 单倍行距
\newcommand{\xiaowu}{\fontsize{9pt}{9pt}\selectfont}		% 小五号


%%%%%%%%%%%%%%%%%%%%%%%%%%%%%%%%%%%%%%%%%%%%%%%%%%%%%%%%%%%
% 重定义一些正文相关标题
%%%%%%%%%%%%%%%%%%%%%%%%%%%%%%%%%%%%%%%%%%%%%%%%%%%%%%%%%%%

%% qiuying comment
%%\theoremstyle{plain} \theorembodyfont{\song\rmfamily}
%%\theoremheaderfont{\hei\rmfamily} \theoremseparator{:}
%%\newtheorem{definition}{\hei 定义}[chapter]
%%\newtheorem{proposition}[definition]{\hei 命题}
%%\newtheorem{lemma}[definition]{\hei 引理}
%%\newtheorem{theorem}{\hei 定理}[chapter]
%%\newtheorem{axiom}{\hei 公理}
%%\newtheorem{corollary}[definition]{\hei 推论}
%%\newtheorem{exercise}[definition]{}
%%
%%\theoremheaderfont{\CJKfamily{hei}\rmfamily}\theorembodyfont{\rmfamily}
%%\theoremstyle{nonumberplain} \theoremseparator{:}
%%\theoremsymbol{$\blacksquare$}
%%\newtheorem{proof}{\hei 证明}
%%
%%\theoremsymbol{$\square$}
%%\newtheorem{example}{\hei 例}
%%

%%%%%%%%%%%%%%%%%%%%%%%%%%%%%%%%%%%%%%%%%%%%%%%%%%%%%%%%%%%%
%% 用于中文段落缩进 和正文版式
%%%%%%%%%%%%%%%%%%%%%%%%%%%%%%%%%%%%%%%%%%%%%%%%%%%%%%%%%%%%
%%\CJKcaption{GB_aloft}
%\xeCJKcaption{gb_452}
%
%\newlength \CJKtwospaces
%
%\def\CJKindent{
%    \settowidth\CJKtwospaces{\CJKchar{"0A1}{"0A1}\CJKchar{"0A1}{"0A1}}%
%    \parindent\CJKtwospaces
%}
%
%
%%\CJKtilde  \CJKindent

%%%%%%%%%%%%%%%%%%%%%%%%%%%%%%%%%%%%%%%%%%%%%%%%%%
%修改脚注样式
%%%%%%%%%%%%%%%%%%%%%%%%%%%%%%%%%%%%%%%%%%%%%%%%%%
\renewcommand{\thefootnote}{\arabic{footnote}}

%%%%%%%%%%%%%%%%%%%%%%%%%%%%%%%%%%%%%%%%%%%%%%%%%%
%定义段落章节的标题和目录项的格式
%%%%%%%%%%%%%%%%%%%%%%%%%%%%%%%%%%%%%%%%%%%%%%%%%%
% Modified By Zhang Ao
% CQUPT Version
\setcounter{secnumdepth}{4}
\setcounter{tocdepth}{2}

\titleformat{\chapter}[hang]
    {\normalfont\erhao\filcenter\song\bf}
    {\erhao{第\CJKnumber{\thechapter}章}}
    {20pt}{\erhao}
\titlespacing{\chapter}{0pt}{-3ex  plus .1ex minus .2ex}{0.25em}

\renewcommand\thesection{\arabic{section}}
\titleformat{\section}[hang]
    {\xiaoer\filcenter\song\vspace{2em}\bf}
    {\xiaoer{第\CJKnumber{\thesection}节}}
    {0.5em}{}{}
\titlespacing{\section}{0pt}{0.5em}{0em}

\renewcommand\thesubsection{\arabic{subsection}}
\titleformat{\subsection}[hang]
    {\sanhao\song\bf}
    {\sanhao{\CJKnumber{\thesubsection}、}}
    {0.5em}{}{}
\titlespacing{\subsection}{0pt}{0.25em}{0em}

\renewcommand\thesubsubsection{\arabic{subsubsection}}
\titleformat{\subsubsection}[hang]
    {\xiaosan\song\bf}
    {\xiaosan{\thesubsubsection、} }
    {0.5em}{}{}
\titlespacing{\subsubsection}{0pt}{0.25em}{0pt}

%Add by Zhang Ao
%CQUPT Version
\renewcommand\contentsname{\erhao目~~~录}
\titlecontents{chapter}[0pt]{\xiaosi\song\vspace{0.2em}}
    {第\CJKnumber{\thecontentslabel}章\quad}{}
    {\hspace{.5em}\titlerule*[5pt]{$\cdot$}\contentspage}
\titlecontents{section}[1em]{\xiaosi\song\vspace{0.2em}}
    {第\CJKnumber{\thecontentslabel}节\quad}{}
    {\hspace{.5em}\titlerule*[5pt]{$\cdot$}\contentspage}
\titlecontents{subsection}[2em]{\xiaosi\song\vspace{0.2em}}
    {\CJKnumber{\thecontentslabel}、}{}
    {\hspace{.5em}\titlerule*[5pt]{$\cdot$}\contentspage}
\titlecontents{subsubsection}[3em]{\xiaosi\song\vspace{0.2em}}
    {{\thecontentslabel、}\quad}{}
    {\hspace{.5em}\titlerule*[5pt]{$\cdot$}\contentspage}

%%%%%%%%%%%%%%%%%%%%%%%%%%%%%%%%%%%%%%%%%%%%%%%%%%%%%%%
% 定义页眉和页脚 使用fancyhdr 宏包
%%%%%%%%%%%%%%%%%%%%%%%%%%%%%%%%%%%%%%%%%%%%%%%%%%%%%%%%

\newcommand{\makeheadrule}{%
    \makebox[0pt][l]{\rule[.7\baselineskip]{\headwidth}{0.8pt}}%
% 1 Line Modified by Lei Wang BaconChina
% XJTU Version
%    \rule[.6\baselineskip]{\headwidth}{0.4pt}\vskip-.8\baselineskip}
% THU Version
    \vskip-.8\baselineskip}
\makeatletter
\renewcommand{\headrule}{
    {\if@fancyplain
    \let\headrulewidth\plainheadrulewidth\fi\makeheadrule}}

%Add by Zhang Ao
%CQUPT Version
\fancyhead{}
\chead{{\xiaowu\bf重庆邮电大学本科毕业设计(论文)}}
\fancyfoot{}
\cfoot{{\wuhao--~\thepage~--}}
\pagestyle{fancyplain}

%%去掉章节标题中的数字
%\renewcommand{\chaptermark}[1]{\markboth{\chaptername \ #1}{}}
%
% \fancyhf{}
%% \fancyfoot[C,C]{\thepage}
%
%%在book文件类别下,\leftmark自动存录各章之章名,\rightmark记录节标题
%
%% Modified by Lei Wang BaconChina
%% XJTU Version
%% \fancyhead[RO]{\CJKfamily{song}\leftmark}
%% \fancyhead[LE]{\CJKfamily{song}西安交通大学博士学位论文}
%% \fancyfoot[C,C]{--~\thepage~--}
%% THU Version
%% \fancyhead[CO]{\CJKfamily{song}\wuhao\leftmark}
%% \fancyhead[CE]{\nwpulogo\fontsize{8pt}{6pt} 西北工业大学~~~ \sanhao\song 本科毕业设计论文}
% \fancyfoot[C,C]{\wuhao \thepage}
%\chead{\sanhao\raisebox{0.04cm}{\nwpulogo 西北工业大学} \song \bfseries{本科毕业设计论文}}

%%%%%%%%%%%%%%%%%%%%%%%%%%%%%%%%%%%%%%%%%%%%%%%%%%%%%%%%
% 设置行距和段落间垂直距离
%%%%%%%%%%%%%%%%%%%%%%%%%%%%%%%%%%%%%%%%%%%%%%%%%%%%%%%%
% 段落之间的竖直距离
\setlength{\parskip}{3pt plus1pt minus1pt}
% 定义行距
%\linestrech{1.6}
\renewcommand{\baselinestretch}{1.2}
%首行缩进
\setlength{\parindent}{2em}

%%%%%%%%%%%%%%%%%%%%%%%%%%%%%%%%%%%%%%%%%%%%%%%%%%%%%%%%
% 调整列表环境的垂直间距
%%%%%%%%%%%%%%%%%%%%%%%%%%%%%%%%%%%%%%%%%%%%%%%%%%%%%%%%
\let\orig@Itemize =\itemize
\let\orig@Enumerate =\enumerate
\let\orig@Description =\description

\def\Myspacing{\itemsep=2pt \topsep=0pt \partopsep=0pt \parskip=0pt \parsep=0pt}

\def\newitemsep{
\renewenvironment{itemize}{\orig@Itemize\Myspacing}{\endlist}
\renewenvironment{enumerate}{\orig@Enumerate\Myspacing}{\endlist}
\renewenvironment{description}{\orig@Description\Myspacing}{\endlist}
}

\def\olditemsep{
\renewenvironment{itemize}{\orig@Itemize}{\endlist}
\renewenvironment{enumerate}{\orig@Enumerate}{\endlist}
\renewenvironment{description}{\orig@Description}{\endlist}
}

\newitemsep

%%%%%%%%%%%%%%%%%%%%%%%%%%%%%%%%%%%%%%%%%%%%%%%%%%%%%%%%
%% 修改引用的格式,
%%%%%%%%%%%%%%%%%%%%%%%%%%%%%%%%%%%%%%%%%%%%%%%%%%%%%%%%
%
%%第一行在引用处数字两边加方框
%%第二行去除参考文献里数字两边的方框
%%\makeatletter
%%\def\@cite#1{\mbox{$\m@th^{\hbox{\@ove@rcfont[#1]}}$}}
%%\renewcommand\@biblabel[1]{#1}
%%\makeatother
%
%% 增加 \ucite 命令使显示的引用为上标形式
%\newcommand{\ucite}[1]{$^{\mbox{\scriptsize \cite{#1}}}$}

%%%%%%%%%%%%%%%%%%%%%%%%%%%%%%%%%%%%%%%%%%%%%%%%%%%%%%%%%%%
% 定制浮动图形和表格标题样式
%%%%%%%%%%%%%%%%%%%%%%%%%%%%%%%%%%%%%%%%%%%%%%%%%%%%%%%%%%%
\renewcommand{\captionfont}{\CJKfamily{song}\rmfamily}
\renewcommand{\captionlabelfont}{\CJKfamily{song}\rmfamily}

% 按清华标准, 去掉图表号后面的:
%\renewcommand{\captionlabeldelim}{\hspace{1em}}

% 按重邮标准, 图表标题字体为五号
\renewcommand{\captionfont}{\wuhao}

%%%%%%%%%%%%%%%%%%%%%%%%%%%%%%%%%%%%%%%%%%%%%%%%%%%%%%%%
%% 定义题头格言的格式
%%%%%%%%%%%%%%%%%%%%%%%%%%%%%%%%%%%%%%%%%%%%%%%%%%%%%%%%
%
%%
%% 用法 \begin{Aphorism}{author}
%%         aphorism
%%      \end{Aphorism}
%
%\newsavebox{\AphorismAuthor}
%\newenvironment{Aphorism}[1]
%{\vspace{0.5cm}\begin{sloppypar} \slshape
%\sbox{\AphorismAuthor}{#1}
%\begin{quote}\small\itshape }
%{\\ \hspace*{\fill}------\hspace{0.2cm} \usebox{\AphorismAuthor}
%\end{quote}
%\end{sloppypar}\vspace{0.5cm}}
%
%%自定义一个空命令,用于注释掉文本中不需要的部分。
%\newcommand{\comment}[1]{}
%
%% This is the flag for longer version
%\newcommand{\longer}[2]{#1}
%
%\newcommand{\ds}{\displaystyle}
%
%% define graph scale
%\def\gs{1.0}
%
%%%%%%%%%%%%%%%%%%%%%%%%%%%%%%%%%%%%%%%%%%%%%%%%%%%%%%%%%%%%%%%%%%%%%%%
%% 自定义项目列表标签及格式 \begin{denselist} 列表项 \end{denselist}
%%%%%%%%%%%%%%%%%%%%%%%%%%%%%%%%%%%%%%%%%%%%%%%%%%%%%%%%%%%%%%%%%%%%%%%
%\newcounter{newlist} %自定义新计数器
%\newenvironment{denselist}[1][可改变的列表题目]{%%%%%定义新环境
%\begin{list}{\textbf{\hei #1} \arabic{newlist}:} %%标签格式
%    {
%    \usecounter{newlist}
%     \setlength{\labelwidth}{22pt} %标签盒子宽度
%     \setlength{\labelsep}{0cm} %标签与列表文本距离
%     \setlength{\leftmargin}{0cm} %左右边界
%     \setlength{\rightmargin}{0cm}
%     \setlength{\parsep}{0ex} %段落间距
%     \setlength{\itemsep}{0ex} %标签间距
%     \setlength{\itemindent}{44pt} %标签缩进量
%     \setlength{\listparindent}{22pt} %段落缩进量
%    }}
%{\end{list}}%%%%%
%
%%添加一些有用的命令
%%Chinese style for the chapter reference. It doesn't work with hyperref
%\newcommand{\chref}[1]{\CJKnumber{\ref{#1}}}
%%adjust Chinese parenthesis space
%\newcommand{\KH}[1]{\!\!(#1)\!\!}
%\newcommand\dlmu@underline[2][5cm]{\hskip1pt\underline{\hb@xt@ #1{\hss#2\hss}}\hskip3pt}
%\let\coverunderline\dlmu@underline
%
%\setlength{\parindent}{2em}
%\renewcommand{\lstlistingname}{\wuhao 源码}
%
%\setlength{\headheight}{24pt}
%
%\newfontfamily\pagella{TeX Gyre Pagella}
%\newfontfamily\monaco{Monaco}
%\newfontfamily\droidmono{Droid Sans Mono}
%
%\lstdefinelanguage{nesc}
%  {morekeywords={components, configuration, event, generic, implementation, includes, interface, module,new, norace, post, provides, signal, task, uses,nx\_struct, nx\_union,command,uint16\_t,uint8\_t,uint32\_t,as,void},sensitive=false,morecomment=[l]{//},morecomment=[s]{/*}{*/},morestring=[b]",}
%
%\lstset{basicstyle=\droidmono\footnotesize,keywordstyle=\color{blue},commentstyle=\color{green},stringstyle=\color{red},tabsize=4,frameround=ffff,escapeinside=``,lineskip=1pt,framerule=0.5pt,xleftmargin=20pt,xrightmargin=10pt,language=nesc,frame=tb,captionpos=b,abovecaptionskip=10pt,numbers=left, framexleftmargin=5mm}
%%\lstset{basicstyle=\droidmono\footnotesize,tabsize=4,frameround=ffff,escapeinside=``,lineskip=1pt,framerule=0.5pt,xleftmargin=20pt,xrightmargin=10pt,language=nesc,frame=tb,captionpos=b,abovecaptionskip=10pt,numbers=left, framexleftmargin=5mm}
%
%\renewcommand\arraystretch{1.25}
